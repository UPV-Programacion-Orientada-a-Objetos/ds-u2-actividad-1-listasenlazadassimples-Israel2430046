\documentclass[12pt, spanish]{article}
\usepackage[utf-8]{inputenc}
\usepackage[spanish]{babel}
\usepackage{geometry}
\usepackage{graphicx}
\usepackage{fancyhdr}
\usepackage{listings}
\usepackage{hyperref}
\usepackage{amsmath}
\usepackage{xcolor}

\geometry{a4paper, margin=2cm}

\lstset{
    language=C++,
    basicstyle=\ttfamily\small,
    keywordstyle=\color{blue}\bfseries,
    commentstyle=\color{gray},
    stringstyle=\color{red},
    showstringspaces=false,
    breaklines=true,
    frame=single,
    numbers=left,
    tabsize=4
}

\pagestyle{fancy}
\fancyhf{}
\rhead{Sistema IoT de Gestión de Sensores}
\lhead{Estructuras de Datos}
\cfoot{\thepage}

\title{
    \textbf{Sistema de Gestión Polimórfica de Sensores para IoT} \\
    \vspace{1cm}
    \Large{Listas Enlazadas Simples Genéricas con Arduino/ESP32}
}

\author{}

\date{Noviembre de 2025}

\begin{document}

\maketitle

\newpage
\tableofcontents
\newpage

\section{Introducción}

Este informe presenta el desarrollo de un \textbf{Sistema de Gestión Polimórfica de Sensores para IoT} que implementa conceptos avanzados de Programación Orientada a Objetos (POO) e Estructuras de Datos en C++.

\subsection{Objetivo General}

Diseñar e implementar un sistema de bajo nivel que gestione múltiples tipos de sensores (temperatura, presión) mediante:
\begin{itemize}
    \item \textbf{Jerarquía Polimórfica}: Clases abstractas y métodos virtuales puros
    \item \textbf{Listas Enlazadas Simples Genéricas}: Plantillas C++ para tipos genéricos
    \item \textbf{Gestión Manual de Memoria}: Implementación de la Regla de los Tres/Cinco
    \item \textbf{Comunicación Serial}: Integración con Arduino/ESP32
\end{itemize}

\subsection{Motivación}

Las limitaciones identificadas en sistemas tradicionales son:
\begin{enumerate}
    \item \textbf{Rigidez de Tipo}: No todos los sensores generan datos del mismo tipo (int vs float)
    \item \textbf{Rigidez de Estructura}: El número de lecturas por sensor es variable
    \item \textbf{Mantenibilidad}: Agregar nuevos tipos de sensores requiere reescribir lógica existente
\end{enumerate}

\section{Manual Técnico}

\subsection{Arquitectura del Sistema}

El sistema está compuesto por los siguientes componentes principales:

\begin{itemize}
    \item \textbf{SensorBase}: Clase abstracta que define la interfaz común
    \item \textbf{SensorTemperatura}: Sensor concreto para lecturas float
    \item \textbf{SensorPresion}: Sensor concreto para lecturas int
    \item \textbf{ListaSensor}: Plantilla genérica para listas enlazadas
    \item \textbf{GestorSensores}: Gestor polimórfico de la lista principal
    \item \textbf{ComunicacionSerial}: Interfaz para Arduino/ESP32
\end{itemize}

\subsection{Diseño de Clases}

\subsubsection{Clase Base Abstracta: SensorBase}

La clase base define la interfaz común para todos los sensores:
\begin{itemize}
    \item Destructor virtual para polimorfismo correcto
    \item Métodos virtuales puros para forzar implementación
    \item Atributo protegido para identificar sensores
\end{itemize}

\subsubsection{Plantilla: ListaSensor}

Implementa una lista enlazada genérica que soporta cualquier tipo de dato:
\begin{itemize}
    \item Inserción al final: O(n)
    \item Búsqueda: O(n)
    \item Obtener mínimo: O(n)
    \item Calcular promedio: O(n)
\end{itemize}

\section{Compilación}

\subsection{Windows}

\begin{verbatim}
mkdir build
cd build
cmake -G "MinGW Makefiles" ..
cmake --build . --config Release
cd ..
build\SistemaIoT.exe
\end{verbatim}

\subsection{Linux/Debian 12}

\begin{verbatim}
mkdir -p build
cd build
cmake -DCMAKE_BUILD_TYPE=Release ..
cmake --build . --config Release
cd ..
./build/SistemaIoT
\end{verbatim}

\section{Resultados}

\subsection{Ejecución del Programa}

El programa genera un menú interactivo con 7 opciones:

\begin{enumerate}
    \item Crear Sensor de Temperatura (FLOAT)
    \item Crear Sensor de Presión (INT)
    \item Registrar Lectura en Sensor
    \item Conectar con Arduino/ESP32
    \item Procesar Lecturas (Polimorfismo)
    \item Ver Estado de Sensores
    \item Cerrar Sistema (Liberar Memoria)
\end{enumerate}

\subsection{Flujo de Ejecución}

\begin{enumerate}
    \item Crear Sensor: Se instancia un SensorTemperatura o SensorPresion
    \item Registrar Lecturas: Se insertan valores en ListaSensor
    \item Procesamiento Polimórfico: Se llama a procesarLectura() en cada sensor
    \item Liberación de Memoria: El destructor virtual limpia recursivamente
\end{enumerate}

\subsection{Análisis de Complejidad}

Las operaciones principales tienen las siguientes complejidades:

\begin{itemize}
    \item Insertar: O(n) - Inserta al final
    \item Buscar: O(n) - Búsqueda lineal
    \item Obtener Mínimo: O(n) - Recorre toda la lista
    \item Eliminar Mínimo: O(n) - Búsqueda más eliminación
    \item Calcular Promedio: O(n) - Suma y división
\end{itemize}

\section{Conclusiones}

\subsection{Logros Alcanzados}

\begin{itemize}
    \item Implementación completa de Listas Enlazadas Simples Genéricas
    \item Jerarquía polimórfica con métodos virtuales puros
    \item Gestión manual de memoria sin fugas
    \item Comunicación bidireccional con Arduino/ESP32
    \item Compatibilidad con Windows y Linux/Debian 12
    \item Generación automática de documentación con Doxygen
    \item Sistema de compilación con CMake
\end{itemize}

\subsection{Conceptos Aplicados}

\begin{enumerate}
    \item POO Avanzada: Herencia, polimorfismo, métodos virtuales puros
    \item Plantillas (Templates): Genéricos en C++
    \item Gestión de Memoria: Punteros, new/delete, destructores virtuales
    \item Estructuras de Datos: Listas enlazadas simples
    \item Comunicación: Puerto serial multiplataforma
\end{enumerate}

\subsection{Mejoras Futuras}

\begin{itemize}
    \item Implementar lista enlazada doble
    \item Agregar sensor de vibración
    \item Base de datos para almacenar historial
    \item Interfaz gráfica con Qt
    \item Protocolo WiFi para ESP32
    \item Logging en archivo
\end{itemize}

\begin{thebibliography}{9}

\bibitem{cpp}
Bjarne Stroustrup. The C++ Programming Language. 4th Edition. Addison-Wesley, 2013.

\bibitem{ds}
Mark Allen Weiss. Data Structures and Algorithm Analysis in C++. 4th Edition. Pearson, 2013.

\bibitem{esp32}
Espressif Systems. ESP32 Technical Reference Manual. Available at: https://docs.espressif.com/

\bibitem{arduino}
Arduino. Arduino Reference. Available at: https://www.arduino.cc/reference/en/

\bibitem{cmake}
CMake. CMake Documentation. Available at: https://cmake.org/documentation/

\end{thebibliography}

\end{document}
